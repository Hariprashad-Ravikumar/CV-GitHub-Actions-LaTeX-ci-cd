\documentclass[11pt]{article}
\usepackage[margin=1in]{geometry}
\usepackage{hyperref}
\usepackage{enumitem}
\usepackage{titlesec}
\usepackage{fancyhdr}
\usepackage{multicol}
\usepackage{parskip}
\usepackage{amsfonts}
\usepackage{tabularx}
\usepackage{comment}



% Section styling
\titleformat{\section}{\large\bfseries}{}{0em}{}

\begin{document}

\begin{center}
    \textbf{\Huge Hariprashad Ravikumar} \\
    \vspace{0.5em}
    PhD Candidate in Physics\\
    New Mexico State University \\
\end{center}

\begin{tabularx}{\textwidth}{@{}Xr@{}}
    \textbf{Website:} \href{https://hariprashad-ravikumar.github.io}{hariprashad-ravikumar.github.io} & \textbf{Email:} \href{mailto:hari1729@nmsu.edu}{hari1729@nmsu.edu}  \\
    \textbf{LinkedIn:} \href{https://www.linkedin.com/in/hariprashad-ravikumar}{linkedin.com/in/hariprashad-ravikumar} & \textbf{Phone:} +1 575-249-9610 
\end{tabularx}
\vspace{-1em}
\begin{tabularx}{\textwidth}{@{}Xr@{}}
\textbf{GitHub:} \href{https://github.com/Hariprashad-Ravikumar}{github.com/Hariprashad-Ravikumar}  \\
\end{tabularx}

\vspace{0.5em}
\section*{Education}
\vspace{-0.7em}
\begin{tabularx}{\textwidth}{@{}lcX@{}}
    \textbf{PhD in Physics} & (Current) & New Mexico State University, USA \\
    \textbf{MS in Physics} & (2024) & New Mexico State University, USA \\
    \textbf{MSc in Physics} & (2021) & National Institute of Technology – Jalandhar, India \\
    \textbf{BSc in Physics} & (2018) &  Dr. N.G.P. Arts and Science College, India
\end{tabularx}

\section*{Technical Skills}
\vspace{-0.5em}
\begin{tabularx}{\textwidth}{@{}lX@{}}
\textbf{Programming Languages:} & Python, C++, Lua, HTML, \LaTeX \\
\textbf{Libraries \& Frameworks:} & NumPy, pandas, Matplotlib, Scikit-learn, PySR, Qiskit \\
\textbf{Scientific Tools:} & CHROMA, MATLAB, Mathematica \\
\textbf{Modeling \& Methods:} & Monte Carlo Simulation, Symbolic Regression, Statistical Modeling\\
\textbf{Tools \& Platforms:} & Git, GitHub, Docker, JupyterLab, Visual Studio, HPC environments, Linux/Unix Shell (Bash) \\
\textbf{Advanced Coursework:} & Graduate-level Computational Physics, Quantum Computing
\end{tabularx}


\section*{PhD Research}
\vspace{-0.3em}
\textbf{Doctoral Advisor:} Dr. Michael Engelhardt (New Mexico State University, USA)
\begin{itemize}
    \item \textbf{Lattice QCD Calculations of TMDs:} Calculating the longitudinal momentum fraction dependence of Transverse Momentum Dependent Parton Distribution Functions (TMDs) using lattice QCD, which is based on Monte Carlo simulations of discretized space-time fields. The analysis also utilizes PySR symbolic regression, a machine learning (ML) technique, to extract analytical functions from the lattice data.
\end{itemize}

\section*{External Research Collaborations}
\vspace{-0.3em}
\textbf{Collaborators:} Dr. Rajan Gupta and Dr. Tanmoy Bhattacharya (Los Alamos National Laboratory, USA)
\begin{itemize}
    \item \textbf{Lattice QCD Calculations of CP Violation Contributions to nEDM:} Conducting lattice QCD calculations of the hadronic matrix elements needed to connect nucleon Electric Dipole Moments (EDMs) to Standard Model (SM) and Beyond Standard Model (BSM) physics. Supported by a Travel Grant from the New Mexico Consortium at Los Alamos.
\end{itemize}

\textbf{Collaborator:} Dr. Chueng-Ryong Ji (North Carolina State University, USA)
\begin{itemize}
    \item \textbf{Interpolating Conformal Algebra Between Instant and Light-Front Forms of Relativistic Dynamics:} Studying the conformal invariance of quantum fields in the interpolating form dynamics, where the interpolation angle parameter spans between the instant form dynamics (IFD) and the light-front dynamics (LFD).
\end{itemize}

\section*{Selected Short-Term Projects}
\vspace{-0.3em}

\begin{itemize}
    \item \textbf{$\mathbb{Z}_2$ Lattice Gauge Monte Carlo Simulation} \hfill \href{https://github.com/Hariprashad-Ravikumar/Z2_LatticeGauge_Monte_Carlo_Simulation}{GitHub Repository} \\
    Developed a Python simulation of $\mathbb{Z}_2$ lattice gauge theory using Markov chain Monte Carlo methods and Metropolis algorithms. Explored confinement phenomena through Wilson loop measurements and benchmarked results against analytical predictions. Served as foundational computational experience in stochastic sampling and lattice QCD simulation ahead of full PhD research.
\end{itemize}

% Certification
\section*{Certifications}
\vspace{-0.3em}
\begin{itemize}
    \item (Apr 2025) \textbf{Google Advanced Data Analytics Professional Certificate} \\
    Credential ID: \href{https://www.coursera.org/account/accomplishments/professional-cert/certificate/U0HU8UKT89L4}{U0HU8UKT89L4} 
    \item (Apr 2025) \textbf{Kaggle Intro to Machine Learning} \\
    \href{https://www.kaggle.com/learn/certification/hariprashadravikumar/intro-to-machine-learning}{View Certificate} 
\end{itemize}

\section*{Selected Talks}
\vspace{-0.3em}
\begin{itemize}
    \item (Jun. 07, 2024) \href{https://indico.cfnssbu.physics.sunysb.edu/event/111/contributions/1001/attachments/335/552/Lattice_QCD_calculations_of_Sivers_TMD_x_dependance____CFNS_school_presentation_Hari__NMSU_Jun_07_2024.pdf}{\textit{"Lattice QCD Calculations of Sivers TMD $x$ Dependency"}}, 2024 CFNS Summer School on the Physics of the Electron-Ion Collider, Center for Frontiers in Nuclear Science, Stony Brook University, NY, USA
    
    \item (May 16, 2024) \href{https://hariprashad-ravikumar.github.io/talks/Lattice_QCD_calculations_of_Sivers_TMD_x_dependance____presentation_Hari__NMSU_May_16_2024.pdf}{\textit{"Lattice QCD Calculations of $x$ Dependence of Sivers TMD"}}, T-2 Seminar, Theoretical Division (T-2), Los Alamos National Laboratory, USA
    
    \item (June 15, 2023) \href{https://indico.jlab.org/event/717/contributions/12720/attachments/9865/14525/Lattice_QCD_calculations_of_TMDs_HUGS_presentation_Hari_NMSU_Jun_15_2023__updated_%20(1).pdf}{\textit{"Lattice QCD Calculations of TMDs"}}, HUGS Student Seminar Presentation, Thomas Jefferson National Accelerator Facility, Newport News, USA
   
    \item (Dec. 02, 2021) \href{https://indico.global/event/13145/contributions/116043/}{\textit{"Interpolating Conformal Algebra Between the Instant Form and the Front Form of Relativistic Dynamics"}}, Light Cone 2021: Physics of Hadrons on the Light Front, Jeju Island, South Korea
\end{itemize}
\noindent\textit{For a full list of talks, please visit:} \href{https://hariprashad-ravikumar.github.io/talks}{hariprashad-ravikumar.github.io/talks}


\section*{MSc Thesis}
\vspace{-0.3em}
Ravikumar, H. (2021, August). \href{https://hariprashad-ravikumar.github.io/publication_pdfs/The%20Poincare%CC%81%20Algebra%20Interpolation%20between%20Instant%20Form%20Dynamics%20(IFD)%20and%20Light%20Front%20Dynamics%20(LFD)%20(Master's%20thesis).pdf}{\textit{The Poincaré algebra interpolation between instant form dynamics (IFD) and light-front dynamics (LFD)}} (Master's thesis). National Institute of Technology, Jalandhar, India.  \\
Supervised by Prof. Harleen Dahiya (NIT Jalandhar) in collaboration with Prof. Chueng-Ryong Ji (North Carolina State University).

\section*{Selected Summer Programs}
\vspace{-0.3em}

\begin{itemize}
    \item \textbf{Jun 03 – Jun 14, 2024:} CFNS Summer School on the Physics of the Electron-Ion Collider, Center for Frontiers in Nuclear Science, Stony Brook University, New York, USA

    \item \textbf{May 30 – Jun 16, 2023:} Hampton University Graduate Studies (HUGS) Summer Program, Thomas Jefferson National Accelerator Facility, Newport News, USA \\
    \textit{Awarded the HUGS Scholarship.}

    \item \textbf{Jan 20 – Jan 26, 2022:} TMD Winter School, Santa Fe, USA

    \item \textbf{Jun 21 – Jun 25, 2021:} National Nuclear Physics Summer School (NNPSS), Universidad Nacional Autónoma de México (Mexico) and Indiana University (USA)
\end{itemize}

\section*{Awards \& Highlights}
\vspace{-0.3em}

\begin{itemize}
    \item Recipient of the \textbf{2023 George and Barbara Goedecke Physics Excellence Fund Scholarship}, awarded by the NMSU Physics Department

    \item Recipient of the \textbf{2021 Graduate Success Scholarship}, awarded by the NMSU Graduate School

    \item Nominated for participation in the \textbf{70th Lindau Nobel Laureate Meeting (2020)}, Germany, by the Department of Science \& Technology, Government of India

    \item Recipient of the \textbf{2018 Indian Academy of Sciences Summer Research Fellowship}
\end{itemize}

\section*{Graduate Assistantships}

\vspace{-0.3em}
\textbf{Research Assistant, NMSU (2022–2025):} \\
Conducting research in Lattice QCD under Dr. Michael Engelhardt, including symbolic regression (PySR), high-performance computing, and theoretical modeling as part of PhD dissertation work.

\vspace{-0.3em}
\textbf{Teaching Assistant, NMSU (2021–2023):} \\
Conducted undergraduate physics labs, led discussion sections, and provided tutoring support for Physics (E\&M and Mechanics) courses.


\end{document}
