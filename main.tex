\documentclass[11pt]{article}
\usepackage[a4paper, total={7in, 10in}]{geometry}
\usepackage{hyperref}
\usepackage{enumitem}
\usepackage{titlesec}
\usepackage{fancyhdr}
\usepackage{multicol}
\usepackage{parskip}
\usepackage{amsfonts}
\usepackage{tabularx}
\usepackage{comment}



\titleformat{\section}{\large\bfseries}{}{0em}{}

\begin{document}

\begin{center}
  \textbf{\Huge Hariprashad Ravikumar} \\[0.6em]
  \vspace{0.5em}
  PhD Candidate in Physics, specializing in HPC \& Machine Learning for High‐Dimensional Data \\
  %New Mexico State University (Las Cruces, NM)\\[0.4em]
%  \textbf{Target location:} San Jose, CA 95123 — relocating Aug 2026
\end{center}

\begin{tabularx}{\textwidth}{@{}Xr@{}}
    \textbf{Website:} \href{https://hariprashad-ravikumar.github.io}{hariprashad-ravikumar.github.io} & \textbf{Email:} \href{mailto:hari1729@nmsu.edu}{hari1729@nmsu.edu}  \\
    \textbf{LinkedIn:} \href{https://www.linkedin.com/in/hariprashad-ravikumar}{linkedin.com/in/hariprashad-ravikumar} & \textbf{Phone:} +1 575-249-9610 
\end{tabularx}
\vspace{-1em}
\begin{tabularx}{\textwidth}{@{}Xr@{}}
\textbf{GitHub:} \href{https://github.com/Hariprashad-Ravikumar}{github.com/Hariprashad-Ravikumar}  \\
\end{tabularx}


\section*{Experience}
\hrule
\vspace{-0.3em}
\textbf{Graduate Research Assistant}, New Mexico State University \hfill (Aug 2021 - Present)
\vspace{-0.2em}

PhD Project: Lattice QCD and Machine Learning Approaches to TMD Physics
\vspace{-0.5em}
\begin{itemize}
    \item Designed and implemented a novel machine learning model using symbolic regression (PySR) to extract interpretable analytical functions from high-dimensional, noisy Monte Carlo simulation data.
    \vspace{-1.5em}
    \item Engineered custom, physics-constrained loss functions to regulate simulation outputs on HPC clusters, enhancing model accuracy by over 93\% and ensuring system stability by enforcing fundamental physical principles.
    \vspace{-0.5em}
    \item Developed high-performance data processing pipelines on HPC clusters to index and process multi-terabyte datasets (30,000+ correlator evaluations), enabling efficient large-scale analysis.
    \vspace{-0.5em}
    \item Leveraged stochastic sampling methods (jackknife, bootstrap) to rigorously quantify model uncertainties, delivering robust and reliable physical observables from simulation outputs.
\end{itemize}


\section*{Independent Collaborations}
\hrule
\vspace{-0.3em}
\begin{enumerate}
    \item \textbf{Los Alamos National Laboratory} - Collaborated with scientists on Computational Physics
    \vspace{-0.7em}
\begin{itemize}
    \item Developed and optimized parallelized C++/CUDA kernels for GPU-accelerated HPC clusters (NERSC Perlmutter), significantly reducing calculation time for multi-terabyte datasets.
    \vspace{-0.5em}
    \item Architected and executed large-scale Monte Carlo simulations, implementing robust statistical methods (Jackknife, chi-squared with covariance) to ensure numerical stability and validate results.
\end{itemize}

\item  \textbf{North Carolina State University} - Collaborated with professor on Mathematical Physics
\vspace{-0.5em}
\begin{itemize}
    \item Implemented and managed Mathematica symbolic computation workflows on HPC clusters (NERSC Perlmutter) to analyze complex algebraic structures and symmetry constraints.
\end{itemize}

\end{enumerate}

\vspace{0.5em}
\section*{Technical Projects}
\hrule
\vspace{-0.3em}
\begin{enumerate}
    \item \textbf{AI-DataScience-Lab: Cloud-Hosted Forecasting App}  
    \hfill \href{https://github.com/Hariprashad-Ravikumar/AI-DataScience-Lab}{GitHub} $|$ \href{https://hariprashad-ravikumar.github.io/AI-DataScience-Lab}{Live App} \\
    \vspace{-2em}
    \begin{itemize}
    \item Built a cloud-hosted ML forecasting platform with automated data ingestion, model training, and retrieval pipelines using AWS/Azure, Flask, and React.
    \vspace{-0.5em}
    \item Integrated GPT API to generate natural-language summaries, bridging structured data with NLP-driven insights.
    \end{itemize}
    

    \item \textbf{Neural Network from Scratch with \texttt{NumPy}}  
    \hfill \href{https://github.com/Hariprashad-Ravikumar/Neural-Network-from-Scratch-with-NumPy}{GitHub} \\
    \vspace{-2em}
    \begin{itemize}
        \item Implemented a two-layer neural network from the ground up in NumPy, building a deep understanding of backpropagation, activation functions (ReLU, softmax), and optimization.
        \vspace{-0.5em}
        \item Trained the model on 5,000 samples from the MNIST dataset, achieving $80\%$ accuracy within 60 epochs by tuning the learning rate.
    \end{itemize}

    \item \textbf{$\mathbb{Z}_2$ Lattice Gauge Monte Carlo Simulation}  
    \hfill \href{https://github.com/Hariprashad-Ravikumar/Z2_LatticeGauge_Monte_Carlo_Simulation}{GitHub} \\
    \vspace{-2em}
    \begin{itemize}
         \item Implemented large-scale Monte Carlo simulations on HPC clusters, validating results against analytical benchmarks and optimizing data processing throughput for large datasets.
    \end{itemize}
\end{enumerate}

%%
\section*{Technical Skills}
\hrule
\vspace{-0.3em}
\begin{tabbing}
\hspace{3.5cm} \= \kill
\textbf{Programming} \> Python, C++, CUDA, Bash, SQL, JavaScript, Lua, HTML/CSS, YAML \\
\textbf{ML \& APIs} \> TensorFlow, PyTorch, Scikit-learn, Pandas, Flask, FastAPI, RAG\\
\textbf{Cloud \& MLOps} \> Azure, AWS (Lambda, S3), CI/CD, Docker, Git, SLURM\\
\textbf{Methods \& HPC} \> MPI, GPU acceleration, Parallel Computing, Regression, Monte Carlo methods
\end{tabbing}

\section*{Education}
\hrule
\vspace{0.3em}

\textbf{PhD in Physics}, New Mexico State University, USA \hfill \textit{Aug 2021 – July. 2026 (expected)} \\
\textbf{MS in Physics}, New Mexico State University, USA \hfill \textit{Aug 2021 – May 2024} \\
\textbf{MSc in Physics}, National Institute of Technology Jalandhar, India \hfill \textit{July 2019 – May 2021} \\
\textbf{BSc in Physics}, Dr. N.G.P. Arts and Science College, India \hfill \textit{June 2015 – May 2018}


% Certification
\section*{Certifications}
\hrule
\vspace{-0.3em}
\begin{itemize}
    \item (Jun 2025) Getting Started with Accelerated Computing in CUDA C/C++ by NVIDIA
    \item (Jun 2025) \href{https://coursera.org/share/b9cffe9c5ba5832ffb99bf7abdd8c384}{Supervised Machine Learning: Regression and Classification by DeepLearning.AI} 
    \item (Apr 2025) \href{https://www.coursera.org/account/accomplishments/professional-cert/certificate/U0HU8UKT89L4}{Google Advanced Data Analytics Professional Certificate} 
\end{itemize}


\section*{Awards}
\hrule
\vspace{-0.3em}

\begin{itemize}
    \item \textbf{2025 NMC Collaboration Grant}, awarded by the New Mexico Consortium to conduct my independent research project in collaboration with scientists at Los Alamos National Laboratory
    \vspace{-0.5em}
    \item \textbf{2023 George and Barbara Goedecke Physics Excellence Fund Scholarship}, awarded by the NMSU Physics Department
    \vspace{-0.5em}
    \item \textbf{2021 Graduate Success Scholarship}, awarded by the NMSU Graduate School
\end{itemize}

\section*{Selected Talks}
\hrule
\vspace{-0.3em}
\begin{itemize}
    \item (Jun 3, 2025) \href{https://hariprashad-ravikumar.github.io/talks/Los_Alamos_T2_talk_First_Principles_Lattice_QCD_Calculations_of_nEDMs__presentation_Hari_NMSU_June_03_2025.pdf}{\textit{"First Principles Lattice QCD Calculations of nEDMs"}}, T-2 Seminar, Theoretical Division, Los Alamos National Laboratory, USA

    
    \item (May 16, 2024) \href{https://hariprashad-ravikumar.github.io/talks/Lattice_QCD_calculations_of_Sivers_TMD_x_dependance____presentation_Hari__NMSU_May_16_2024.pdf}{\textit{"Lattice QCD Calculations of $x$ Dependence of Sivers TMD"}}, T-2 Seminar, Theoretical Division, Los Alamos National Laboratory, USA
    
\end{itemize}

\noindent\textit{Full list available at:} \href{https://hariprashad-ravikumar.github.io/talks}{hariprashad-ravikumar.github.io/talks}


\section*{Volunteering}
\hrule
\vspace{-0.3em}
\begin{itemize}
    \item \textbf{Vice President}, Physics Graduate Student Organization (NMSU)  \hfill Sep 2025 -- Present \\ \textit{Organized professional development events and served as the primary liaison between graduate students and faculty.}
    
\end{itemize}



\end{document}
