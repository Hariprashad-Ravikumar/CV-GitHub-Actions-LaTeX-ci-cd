\documentclass[11pt]{article}
\usepackage[a4paper, total={7.5in, 10.9in}]{geometry}
\usepackage{hyperref}
\usepackage{enumitem}
\usepackage{titlesec}
\usepackage{fancyhdr}
\usepackage{multicol}
\usepackage{parskip}
\usepackage{amsfonts}
\usepackage{tabularx}
\usepackage{comment}



\titleformat{\section}{\large\bfseries}{}{0em}{}

\begin{document}

\begin{center}
  \textbf{\Huge Hariprashad Ravikumar} \\[0.6em]
  \vspace{0.5em}
  PhD Candidate specializing in High-Performance Computing, Deep Learning, and AI for Science\\
  Expertise in GPU-accelerated computing with C++/CUDA and Python
  %PhD Candidate specializing in AI for Physics-Based Simulation and High-Performance Computing (HPC)\\
  %Expertise in GPU-accelerated computing with C++/CUDA and Deep Learning (PyTorch/TensorFlow) 
  %Expertise in Statistical Modeling and Deep Learning (PyTorch/TensorFlow)
 \\
  %New Mexico State University (Las Cruces, NM)\\[0.4em]
%  \textbf{Target location:} San Jose, CA 95123 — relocating Aug 2026
\end{center}

\begin{tabularx}{\textwidth}{@{}Xr@{}}
    \textbf{Website:} \href{https://hariprashad-ravikumar.github.io}{hariprashad-ravikumar.github.io} & \textbf{Email:} \href{mailto:hari1729@nmsu.edu}{hari1729@nmsu.edu}  \\
    \textbf{LinkedIn:} \href{https://www.linkedin.com/in/hariprashad-ravikumar}{linkedin.com/in/hariprashad-ravikumar} & \textbf{Phone:} +1 575-249-9610 
\end{tabularx}
\vspace{-1em}
\begin{tabularx}{\textwidth}{@{}Xr@{}}
\textbf{GitHub:} \href{https://github.com/Hariprashad-Ravikumar}{github.com/Hariprashad-Ravikumar}  \\
\end{tabularx}


\section*{Experience}
\hrule
\vspace{-0.3em}
\textbf{Graduate Research Assistant}, New Mexico State University \hfill \textit{(Aug 2021 - Present)}
\vspace{-0.2em}

PhD Project: Lattice QCD and Machine Learning Approaches to TMD Physics
\vspace{-0.5em}
\begin{itemize}
\item Generated 30,000+ high-fidelity synthetic data points by solving Partial Differential Equations with large-scale Monte Carlo simulations and built an end-to-end AI for Science pipeline to model the underlying physics, achieving over 98\% predictive accuracy with symbolic regression machine learning.
\vspace{-0.5em}
\item Developed GPU-accelerated CUDA/C++ pipelines for large-scale FFT-based feature extraction, reducing data processing time by 10x
\vspace{-0.5em}
\item Developed production-grade Python, Lua, \& Mathematica packages for reproducible statistical analysis, ensuring numerical stability in multi-stage workflows (jackknife resampling, uncertainty propagation)
\end{itemize}


\section*{Independent Collaborations}
\hrule
\vspace{-0.3em}
\begin{enumerate}
    \item \textbf{Los Alamos National Laboratory} - Computational Physics \hfill \textit{(May 2024 - Present)}
\vspace{-0.5em}
\begin{itemize}
    \item Accelerated multi-terabyte scientific calculations by developing and optimizing parallelized C++ CUDA kernels for GPU-accelerated HPC clusters (NERSC Perlmutter), significantly reducing runtime for large-scale Monte Carlo simulations.
    \vspace{-0.5em}
    \item Managed and executed 75,000+ CPU/GPU compute hours by designing and deploying custom SLURM workflows for large-scale job orchestration
    \vspace{-0.5em}
    %\item Investigated advanced simulation techniques using gradient flow, a method conceptually similar to Flow-Based Generative Models, to analyze the properties of quantum systems and ensure numerical stability
    \item Investigated advanced simulation techniques using gradient flow, a method conceptually similar to Flow-Based Generative Models, to enhance simulation fidelity and ensure numerical stability
    \vspace{-0.4em}
    \item Increased model reliability through rigorous statistical validation on over 50,000 correlated data points, applying methods like AIC-based selection and chi-squared minimization with full covariance matrices.
\end{itemize}

    \item  \textbf{North Carolina State University} - Mathematical Physics  \hfill \textit{(Dec 2020 - Present)}
\vspace{-0.5em}
\begin{itemize}
    \item Implemented and managed Mathematica symbolic computation workflows on HPC clusters to analyze complex algebraic structures and symmetry constraints.
\end{itemize}

\end{enumerate}

\section*{Publications}
\hrule
\vspace{-0.3em}
\begin{enumerate}[leftmargin=1.5em]
    \item C.-R. Ji, \textbf{H. Ravikumar}, \textit{"Interpolating conformal algebra in (1+1) dimensions,"} \textbf{Physical Review D} (Submitted, 2026). Preprint: \href{https://arxiv.org/abs/2601.19251}{arXiv:2601.19251 [hep-th]}
    
    %\item \textbf{H. Ravikumar} and C.-R. Ji, \textit{"Interpolating conformal algebra (3+1) between the instant form and the front form of relativistic dynamics,"} (In Preparation).
\end{enumerate}

\vspace{0.5em}
\section*{Technical Projects}
\hrule
\vspace{-0.3em}
\begin{enumerate}
    \item \textbf{AI-DataScience-Lab: Full-Stack Data Product}  
    \hfill \href{https://github.com/Hariprashad-Ravikumar/AI-DataScience-Lab}{GitHub} $|$ \href{https://hariprashad-ravikumar.github.io/AI-DataScience-Lab}{Live App} \\
    \vspace{-2em}
    \begin{itemize}
    \item Prototyped and deployed a scalable full-stack data tool (Python, Flask, Azure) to deliver on-demand data-driven insights and actionable recommendations, integrating an OpenAI API to demonstrate automated data storytelling and data interpretation.
    \vspace{-0.5em}
    \item Implemented an end-to-end statistical analysis pipeline using Python (Pandas) for data cleaning, statistical modeling (Scikit-learn) for regression, and Matplotlib for data visualization.
    \end{itemize}

    %\item \textbf{$\mathbb{Z}_2$ Lattice Gauge Monte Carlo Simulation}  
    \item \textbf{Physics Based Monte Carlo Simulation}
    \hfill \href{https://github.com/Hariprashad-Ravikumar/Z2_LatticeGauge_Monte_Carlo_Simulation}{GitHub} \\
    \vspace{-2em}
    \begin{itemize}
         \item Developed a Physics-Based Simulation from scratch to generate synthetic lattice gauge configurations using Monte Carlo methods on HPC clusters, validating the generated data against known analytical benchmarks.
    \end{itemize}
    
    %\item \textbf{AI-DataScience-Lab: Cloud-Hosted Forecasting App}  
    %\hfill \href{https://github.com/Hariprashad-Ravikumar/AI-DataScience-Lab}{GitHub} $|$ \href{https://hariprashad-ravikumar.github.io/AI-DataScience-Lab}{Live App} \\
    %\vspace{-2em}
    %\begin{itemize}
    %\item Developed a full-stack ML forecasting platform on AWS/Azure featuring automated MLOps pipelines and a GPT API for generating natural-language insights.
    %\end{itemize}
    \vspace{2em}
    \item \textbf{Neural Network from Scratch with \texttt{NumPy}}  
    \hfill \href{https://github.com/Hariprashad-Ravikumar/Neural-Network-from-Scratch-with-NumPy}{GitHub} \\
    \vspace{-2em}
    \begin{itemize}
        \item Implemented and trained a neural network from scratch in NumPy for MNIST digit recognition, achieving 80\% accuracy by building and tuning core components like backpropagation and activation functions
    \end{itemize}
\end{enumerate}

%%
\section*{Technical Skills}
\hrule
\vspace{-0.3em}
\begin{tabbing}
\hspace{3.5cm} \= \kill
\textbf{Programming} \> Python, C++, CUDA, Bash, SQL, Lua, HTML/CSS, YAML \\
\textbf{ML \& APIs} \> Numba, TensorFlow, PyTorch, Scikit-learn, Pandas, cuFFT, cuDNN, Flask, FastAPI, RAG\\
\textbf{Cloud \& MLOps} \> Azure, AWS (Lambda, S3), CI/CD, Docker, Git, SLURM\\
\textbf{Methods \& HPC} \> Parallel Computing (GPU, MPI), Numerical Methods (PDEs, Monte Carlo, Regression)
\end{tabbing}

\section*{Education}
\hrule
\vspace{0.3em}

\textbf{PhD in Physics}, New Mexico State University, USA \hfill \textit{Aug 2021 – Dec 2026 (expected)} \\
%\textit{(Relevant Coursework: Advanced Computational Physics, Statistical Mechanics, Quantum Computing)}\\
\textbf{MS in Physics}, New Mexico State University, USA \hfill \textit{Aug 2021 – May 2024} \\
\textbf{MSc in Physics}, National Institute of Technology Jalandhar, India \hfill \textit{July 2019 – May 2021} \\
\textbf{BSc in Physics}, Dr. N.G.P. Arts and Science College, India \hfill \textit{June 2015 – May 2018}

\vspace{-1em}
% Certification
\section*{Certifications}
\hrule
\vspace{-0.3em}
\begin{itemize}
    \item Getting Started with Accelerated Computing in CUDA C/C++ by NVIDIA
    \item \href{https://learn.nvidia.com/certificates?id=mMWLgny_SEC5DgHXY9XYEw}{Fundamentals of Accelerated Computing with CUDA Python by NVIDIA} 
    \item \href{https://www.coursera.org/account/accomplishments/verify/XG3YT41S0PF5}{Advanced Learning Algorithms by DeepLearning.AI} 
    \item \href{https://coursera.org/share/b9cffe9c5ba5832ffb99bf7abdd8c384}{Supervised Machine Learning: Regression and Classification by DeepLearning.AI} 
    \item \href{https://www.coursera.org/account/accomplishments/professional-cert/certificate/U0HU8UKT89L4}{Google Advanced Data Analytics Professional Certificate} 
\end{itemize}


\section*{Awards}
\hrule
\vspace{-0.3em}

\begin{itemize}
    \item \textbf{2025 NMC Collaboration Grant}, awarded by the New Mexico Consortium to conduct my independent research project in collaboration with scientists at Los Alamos National Laboratory
    \vspace{-0.5em}
    \item \textbf{2023 George and Barbara Goedecke Physics Excellence Fund Scholarship}, awarded by the NMSU Physics Department
    \vspace{-0.5em}
   %\item \textbf{2021 Graduate Success Scholarship}, awarded by the NMSU Graduate School
\end{itemize}

\section*{Selected Talks}
\hrule
\vspace{-0.3em}
\begin{itemize}
    \item (Jun 3, 2025) \href{https://hariprashad-ravikumar.github.io/talks/Los_Alamos_T2_talk_First_Principles_Lattice_QCD_Calculations_of_nEDMs__presentation_Hari_NMSU_June_03_2025.pdf}{\textit{"First Principles Lattice QCD Calculations of nEDMs"}}, T-2 Seminar, Theoretical Division, \textbf{Los Alamos National Laboratory}, USA

    %\item (Jun 07, 2025) \href{https://indico.cfnssbu.physics.sunysb.edu/event/111/contributions/1001/attachments/335/552/Lattice_QCD_calculations_of_Sivers_TMD_x_dependance____CFNS_school_presentation_Hari__NMSU_Jun_07_2024.pdf}{\textit{"Lattice QCD calculations of Sivers TMD $x$ dependency"}}, Invited talk at the CFNS Summer School, \textbf{Stony Brook University}, USA

    \item (May 16, 2024) \href{https://hariprashad-ravikumar.github.io/talks/Lattice_QCD_calculations_of_Sivers_TMD_x_dependance____presentation_Hari__NMSU_May_16_2024.pdf}{\textit{"Lattice QCD Calculations of $x$ Dependence of Sivers TMD"}}, T-2 Seminar, Theoretical Division, \textbf{Los Alamos National Laboratory}, USA

    \item (June 15, 2023) \href{https://indico.jlab.org/event/717/contributions/12720/attachments/9865/14525/Lattice_QCD_calculations_of_TMDs_HUGS_presentation_Hari_NMSU_Jun_15_2023__updated_%20(1).pdf}{\textit{"Lattice QCD calculations of TMDs"}}, HUGS Student Seminar, \textbf{Thomas Jefferson National Accelerator Facility}, USA
\end{itemize}

\noindent\textit{Full list available at:} \href{https://hariprashad-ravikumar.github.io/talks}{hariprashad-ravikumar.github.io/talks}


\section*{Volunteering}
\hrule
\vspace{-0.3em}
\begin{itemize}
    \item \textbf{Vice President}, Physics Graduate Student Organization (NMSU)  \hfill \textit{Aug 2025 -- Present} \\ Organized professional development events and served as the primary liaison between 40+ graduate students and faculty.
    
\end{itemize}

%\section*{Relevant Graduate Coursework}
%\hrule
%\vspace{-0.3em}
%\begin{itemize}
%    \item Advanced Computational Physics, Statistical Mechanics, %Quantum Computing
%\end{itemize}



\end{document}
